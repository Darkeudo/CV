\documentclass[11pt,a4paper]{article}
\usepackage[margin=1in]{geometry}
\usepackage{titlesec}
\usepackage{enumitem}
\usepackage{hyperref}
\usepackage{parskip}
\usepackage{xcolor}

\titleformat{\section}{\large\bfseries\color{black}}{}{0em}{}[\titlerule]
\titleformat{\subsection}[runin]{\bfseries}{}{0em}{}[]
\renewcommand{\labelitemi}{$\cdot$}

\begin{document}

\begin{center}
  {\LARGE \textbf{Eudomar Gómez Díaz}} \\
  \vspace{2pt}
  Desarrollador de Software \\
  \href{mailto:eudomargomezdiaz@gmail.com}{eudomargomezdiaz@gmail.com} \quad | \quad +57 310 482 0882 \\
   Sogamoso / Boyacá – Colombia
\end{center}

\vspace{10pt}

\section*{Perfil Profesional}
Desarrollador de software con más de 4 años de experiencia combinada en análisis, diseño, desarrollo y soporte de aplicaciones web. Con enfoque en la mejora continua, estándares de calidad, interacción con stakeholders y adaptación a entornos híbridos y remotos. Manejo de tecnologías modernas backend, gestión de infraestructura y cumplimiento de normativas ISO.

\section*{Experiencia Profesional}

\subsection*{Desarrollador de Software \hfill Cadintek — Bogotá / Remoto (Jun. 2023 – Jul. 2025)}
\begin{itemize}[leftmargin=*]
  \item Diseño y desarrollo de aplicaciones web siguiendo estándares tecnológicos y mejores prácticas de la industria.
  \item Interacción directa con stakeholders durante el ciclo de desarrollo para definición y validación de requisitos.
  \item Soporte técnico de nivel profesional en resolución de incidencias y mejoras evolutivas.
  \item Desarrollo, validación y mantenimiento de código limpio, documentado y alineado a especificaciones técnicas.
  \item Participación activa en procesos de actualización tecnológica y adopción de nuevas herramientas.
\end{itemize}

\subsection*{Ingeniero de Software \hfill CDT Mineral Accredited Lab — Sogamoso / Híbrido (Mar. 2021 – Actualidad)}
\begin{itemize}[leftmargin=*]
  \item Soporte de tercer nivel en infraestructura TI (física, en la nube y on-premise).
  \item Diseño y desarrollo de plataforma web para gestión de laboratorio bajo norma ISO 17025:2017.
  \item Implementación de buenas prácticas de seguridad para confidencialidad e integridad de la información.
  \item Automatización de procesos de laboratorio y digitalización de formularios técnicos.
\end{itemize}

\section*{Educación}

\textbf{Tecnólogo en Análisis y Desarrollo de Software} \\
Servicio Nacional de Aprendizaje – SENA \hfill (Sept. 2023 – Dic. 2025)

\textbf{Administrador de Empresas} \\
Universidad Pedagógica y Tecnológica de Colombia \hfill (Ene. 2011 – Dic. 2017)

\textbf{Certificación: Reparación de Equipos de Cómputo} \\
SENA \hfill (Nov. 2020) — Certificado de Competencia Laboral

\section*{Habilidades Técnicas}
\begin{tabular}{ll}
\textbf{Lenguajes:} & Go, JavaScript, PHP, Python \\
\textbf{Frameworks:} & Angular, Frameworks personalizados (PHP, Go) \\
\textbf{CMS:} & Drupal (desarrollo de módulos y temas personalizados) \\
\textbf{Arquitectura:} & Diseño de APIs RESTful, autenticación JWT, MVC \\
\textbf{Base de datos:} & PostgreSQL, MySQL, Redis \\
\textbf{DevOps / Cloud:} & Linux, Nginx, Apache2, Git, Docker, Infraestructura híbrida \\
\textbf{Metodologías:} & Ágiles (Scrum), Modelo en V \\
\textbf{Otros:} & Análisis de requisitos, soporte técnico, normas ISO \\
\end{tabular}

\section*{Idiomas}
Español (nativo), Inglés (Básico)

% \section*{Referencias}
% Disponibles bajo solicitud.

\end{document}
